\documentclass[11pt]{article}

\usepackage{amsmath,amssymb}
\usepackage{geometry}
\geometry{margin=1in}

\title{\textbf{Equations, Variables, and Functional Role\\
of the RFI Model}}
\author{Utkarsh Maurya}
\date{}

\begin{document}
\maketitle


%------------------------------------------------
\section{Propagation and Link Budget}

\subsection{Free-Space Path Loss (ITU-R P.525)}
\begin{equation}
L_{\mathrm{fs}}(f,d) =
32.45 + 20\log_{10}(f_{\mathrm{MHz}})
+ 20\log_{10}(d_{\mathrm{km}})
\end{equation}

\begin{itemize}
\item \textbf{Used for:} Modeling geometric spreading loss between a transmitter
and a receiver for both desired and interfering signals.
\item \textbf{Provides:} Frequency- and distance-dependent attenuation used to
convert transmitter EIRP into received power.
\item \textbf{Variables:}
\begin{itemize}
    \item $L_{\mathrm{fs}}$: free-space path loss [dB]
    \item $f_{\mathrm{MHz}}$: carrier frequency [MHz]
    \item $d_{\mathrm{km}}$: propagation distance [km]
\end{itemize}
\end{itemize}

\subsection{Received Carrier Power}
\begin{equation}
C = EIRP_{\mathrm{tx}}(f)
+ G_{\mathrm{rx}}(f,\theta_0)
- L_{\mathrm{path}}(f,d_0)
- L_{\mathrm{other}}
\end{equation}

\begin{itemize}
\item \textbf{Used for:} Establishing the interference-free reference level of the
victim communication link.
\item \textbf{Provides:} Received carrier power used as the baseline for all
interference and SNR degradation metrics.
\item \textbf{Variables:}
\begin{itemize}
    \item $C$: received carrier power [dBW]
    \item $EIRP_{\mathrm{tx}}$: transmit equivalent isotropically radiated power [dBW]
    \item $G_{\mathrm{rx}}$: receive antenna gain toward the desired signal [dB]
    \item $L_{\mathrm{path}}$: propagation losses (free-space and fixed atmospheric) [dB]
    \item $L_{\mathrm{other}}$: fixed additional losses (polarization, implementation) [dB]
\end{itemize}
\end{itemize}

%------------------------------------------------
\section{Antenna Pattern Model}

\subsection{Receive Antenna Off-Axis Gain (ITU-R S.1528-type)}

Main-lobe region:
\begin{equation}
G_{\mathrm{rx}}(\theta) =
G_{\max}
- 12\left(\frac{\theta}{\theta_{3\mathrm{dB}}}\right)^2
\end{equation}

Side-lobe region:
\begin{equation}
G_{\mathrm{rx}}(\theta) =
G_{\max} - 30\ \mathrm{dB}
\end{equation}

\begin{itemize}
\item \textbf{Used for:} Modeling angular discrimination of the receive antenna for
off-axis interfering signals.
\item \textbf{Provides:} Receive antenna gain as a function of arrival angle.
\item \textbf{Variables:}
\begin{itemize}
    \item $G_{\mathrm{rx}}(\theta)$: receive antenna gain at off-axis angle $\theta$ [dB]
    \item $G_{\max}$: maximum (boresight) antenna gain [dB]
    \item $\theta$: off-axis angle relative to antenna boresight [deg]
    \item $\theta_{3\mathrm{dB}}$: 3-dB beamwidth of the antenna [deg]
\end{itemize}
\end{itemize}

%------------------------------------------------
\section{Interference Modeling}

\subsection{Single-Entry Interference Power}
\begin{equation}
I_i =
EIRP_i(f)
- L_{\mathrm{fs}}(f,d_i)
- L_{\mathrm{atm}}(f,d_i,W_i)
+ G_{\mathrm{rx}}(f,\theta_i)
- L_{\mathrm{misc}}
\end{equation}

\begin{itemize}
\item \textbf{Used for:} Computing the received interference contribution from a
single interfering transmitter.
\item \textbf{Provides:} Interference power at the victim receiver from interferer $i$.
\item \textbf{Variables:}
\begin{itemize}
    \item $I_i$: interference power from source $i$ [dBW]
    \item $EIRP_i$: equivalent isotropically radiated power of interferer $i$ toward the victim [dBW]
    \item $d_i$: distance between interferer $i$ and victim receiver [km]
    \item $\theta_i$: off-axis angle of interferer $i$ at the victim antenna [deg]
    \item $L_{\mathrm{atm}}$: atmospheric or weather-dependent attenuation [dB]
    \item $L_{\mathrm{misc}}$: fixed miscellaneous losses [dB]
\end{itemize}
\end{itemize}

\subsection{Aggregate Interference Power}
\begin{equation}
I_{\mathrm{agg}} =
10\log_{10}
\left(
\sum_i 10^{I_i/10}
\right)
\end{equation}

\begin{itemize}
\item \textbf{Used for:} Combining multiple simultaneous interference sources into
a single equivalent interference level.
\item \textbf{Provides:} Total aggregate interference power consistent with ITU-R
aggregate-interference methodology.
\item \textbf{Variables:}
\begin{itemize}
    \item $I_{\mathrm{agg}}$: aggregate interference power [dBW]
    \item $I_i$: individual interference powers [dBW]
\end{itemize}
\end{itemize}

\subsection{Carrier-to-Interference Ratio}
\begin{equation}
\frac{C}{I} = C - I_{\mathrm{agg}}
\end{equation}

\begin{itemize}
\item \textbf{Used for:} Assessing the dominance of the desired signal over aggregate interference.
\item \textbf{Provides:} Carrier-to-interference ratio used for robustness comparison.
\item \textbf{Variables:}
\begin{itemize}
    \item $C/I$: carrier-to-interference ratio [dB]
\end{itemize}
\end{itemize}

%------------------------------------------------
\section{Noise and Signal-to-Noise Ratio}

\subsection{Thermal Noise Power}
\begin{equation}
N =
10\log_{10}
\left(
k_{\mathrm{B}} T_{\mathrm{sys}} B
\right)
\end{equation}

\begin{itemize}
\item \textbf{Used for:} Modeling the intrinsic noise floor of the receiver.
\item \textbf{Provides:} Thermal noise power at the receiver input.
\item \textbf{Variables:}
\begin{itemize}
    \item $N$: thermal noise power [dBW]
    \item $k_{\mathrm{B}}$: Boltzmann constant [J/K]
    \item $T_{\mathrm{sys}}$: system noise temperature [K]
    \item $B$: receiver bandwidth [Hz]
\end{itemize}
\end{itemize}

\subsection{Interference-Free Signal-to-Noise Ratio}
\begin{equation}
\mathrm{SNR}_0 = C - N
\end{equation}

\begin{itemize}
\item \textbf{Used for:} Defining the baseline link margin without interference.
\item \textbf{Provides:} Reference SNR for evaluating interference-induced degradation.
\item \textbf{Variables:}
\begin{itemize}
    \item $\mathrm{SNR}_0$: interference-free signal-to-noise ratio [dB]
\end{itemize}
\end{itemize}

\subsection{Signal-to-Noise Ratio with Interference}
\begin{equation}
\mathrm{SNR}_{\mathrm{I}} =
10 \log_{10}
\left(
\frac{10^{C/10}}
{10^{N/10} + 10^{I_{\mathrm{agg}}/10}}
\right)
\end{equation}

\begin{itemize}
\item \textbf{Used for:} Evaluating link performance in the presence of interference.
\item \textbf{Provides:} Operational signal-to-noise ratio under interference conditions.
\item \textbf{Variables:}
\begin{itemize}
    \item $\mathrm{SNR}_{\mathrm{I}}$: signal-to-noise ratio with interference [dB]
\end{itemize}
\end{itemize}

%------------------------------------------------
\section{RFI-Induced Attenuation}

\subsection{SNR Loss Due to Interference}
\begin{equation}
\Delta \mathrm{SNR} =
\mathrm{SNR}_0 - \mathrm{SNR}_{\mathrm{I}}
\end{equation}

\begin{itemize}
\item \textbf{Used for:} Isolating degradation caused solely by RFI.
\item \textbf{Provides:} SNR loss used for probabilistic ITU-R compliance evaluation.
\item \textbf{Variables:}
\begin{itemize}
    \item $\Delta \mathrm{SNR}$: RFI-induced attenuation [dB]
\end{itemize}
\end{itemize}

%------------------------------------------------
\section{Equivalent Power Flux Density}

\subsection{Equivalent Power Flux Density (EPFD)}
\begin{equation}
\mathrm{EPFD} =
EIRP_i
- L_{\mathrm{fs}}(f,d_{\mathrm{km}})
+ G_{\mathrm{rx}}
- 10\log_{10}(B_{\mathrm{MHz}})
\end{equation}



\begin{itemize}
\item \textbf{Used for:} Expressing interference in a receiver-independent metric
for regulatory and satellite compatibility analysis.
\item \textbf{Provides:} Equivalent power flux density per unit bandwidth, directly
comparable to ITU-R EPFD protection limits.
\item \textbf{Variables:}
\begin{itemize}
    \item $\mathrm{EPFD}$: equivalent power flux density [dBW/m$^2$/MHz]
    \item $EIRP_i$: equivalent isotropically radiated power of interferer $i$ [dBW]
    \item $L_{\mathrm{fs}}(f,d_{\mathrm{km}})$: free-space path loss per ITU-R P.525 [dB]
    \item $d_{\mathrm{km}}$: interferer-to-victim distance [km]
    \item $G_{\mathrm{rx}}$: receive antenna gain in the direction of the interferer [dB]
    \item $B_{\mathrm{MHz}}$: reference bandwidth [MHz]
\end{itemize}
\end{itemize}

This formulation expresses EPFD using a link-budget approach consistent with the
rest of the model; geometric spreading is fully captured by the free-space path
loss term $L_{\mathrm{fs}}$ defined in ITU-R P.525.

%------------------------------------------------
\section{LEO--GEO Orbital Geometry}

\subsection{Orbital Angular Velocity (Circular Orbit)}
\begin{equation}
\omega = \sqrt{\frac{\mu}{r^3}}, \qquad r = R_{\mathrm{E}} + h_{\mathrm{LEO}}
\end{equation}

\subsection{LEO Position (Equatorial Plane)}
\begin{equation}
x_{\mathrm{LEO}} = r \cos(\omega t + \phi_0), \qquad
y_{\mathrm{LEO}} = r \sin(\omega t + \phi_0)
\end{equation}
GEO fixed at $(x_{\mathrm{GEO}}, y_{\mathrm{GEO}}) = (R_{\mathrm{E}} + h_{\mathrm{GEO}}, 0)$.

\subsection{Slant Range}
\begin{equation}
d_{\mathrm{slant}} = \sqrt{(x_{\mathrm{LEO}} - x_{\mathrm{GEO}})^2 + (y_{\mathrm{LEO}} - y_{\mathrm{GEO}})^2}
\end{equation}

\subsection{Off-Axis Angle at GEO}
Angle between GEO--Earth-center vector and GEO--LEO vector; used with S.1528 for $G_{\mathrm{rx}}(\theta)$.

\begin{itemize}
\item \textbf{Variables:} $\mu$: gravitational parameter [km$^3$/s$^2$], $R_{\mathrm{E}}$: Earth radius [km], $h_{\mathrm{LEO}}$, $h_{\mathrm{GEO}}$: altitudes [km], $\phi_0$: initial phase [rad].
\end{itemize}

%------------------------------------------------
\section{Rain Attenuation}

\subsection{Specific Attenuation}
\begin{equation}
\gamma = k \cdot R^{\alpha}
\end{equation}
Frequency-dependent coefficients (simplified): $f < 5$\,GHz: $k=0.0001$, $\alpha=1.0$; $5$--$15$\,GHz: $k=0.01$, $\alpha=1.1$; $15$--$25$\,GHz: $k=0.05$, $\alpha=1.2$; $f \geq 25$\,GHz: $k=0.15$, $\alpha=1.3$.

\subsection{Rain Attenuation on Path}
\begin{equation}
A_{\mathrm{rain}} = \gamma \cdot L_{\mathrm{eff}}
\end{equation}

\begin{itemize}
\item \textbf{Variables:} $\gamma$: specific attenuation [dB/km], $R$: rain rate [mm/h], $L_{\mathrm{eff}}$: effective path length [km].
\end{itemize}

%------------------------------------------------
\section{Link Adaptation}

\subsection{Modulation Table (AMC)}
Spectral efficiency $\eta$ selected by SNR threshold: QPSK $\geq 0$\,dB $\to$ 2\,bps/Hz; 8PSK $\geq 5$\,dB $\to$ 3\,bps/Hz; 16APSK $\geq 10$\,dB $\to$ 4\,bps/Hz; 32APSK $\geq 15$\,dB $\to$ 5\,bps/Hz. Below lowest threshold: $\eta = 0$ (outage).

\subsection{Throughput}
\begin{equation}
\text{Throughput} = \eta \cdot B
\end{equation}
$\eta$ in bps/Hz, $B$ in Hz; result in bps.

\subsection{Throughput Degradation}
\begin{equation}
\text{Degradation (\%)} = 100 \cdot \frac{\mathrm{mean}(P_{\mathrm{base}} - P_{\mathrm{deg}})}{\mathrm{mean}(P_{\mathrm{base}})}
\end{equation}
$P_{\mathrm{base}}$: baseline throughput time series; $P_{\mathrm{deg}}$: degraded (or joint) throughput time series.

%------------------------------------------------
\section{RFI Robustness Index (RRI)}

\subsection{Definition}
\begin{equation}
\mathrm{RRI} =
w_{\mathrm{t}}\left(1 - \frac{T}{100}\right)
+ w_{\mathrm{a}} \frac{A}{100}
+ w_{\mathrm{e}}\left(1 - \frac{E}{100}\right)
+ w_{\mathrm{o}}\left(1 - \frac{O}{100}\right)
\end{equation}
Default equal weights: $w_{\mathrm{t}} = w_{\mathrm{a}} = w_{\mathrm{e}} = w_{\mathrm{o}} = 0.25$.

\begin{itemize}
\item \textbf{Variables:} $T$: throughput degradation [\%]; $A$: link availability [\%]; $E$: EPFD exceedance [\%]; $O$: joint outage [\%]. Higher RRI corresponds to more robust link.
\end{itemize}


\end{document}
